
\documentclass[11pt]{article}
\usepackage[margin=1in]{geometry}
\usepackage{longtable}
\usepackage{array}
\usepackage{hyperref}
\usepackage{titlesec}

\titleformat{\section}{\large\bfseries}{}{0em}{}

%\title{\textbf{Machine Learning in Astronomy}}
%\date{}

\begin{document}

%\maketitle

\section*{Machine Learning in Astronomy}

A study of machine learning applications in sky surveys, reproducible research, and open scientific collaboration.

This organization develops methods and infrastructure for applying modern machine learning techniques to large-scale astronomical survey data, including imaging, catalogs, and cross-matched multi-survey datasets.

\subsection*{Papers \& Presentations}

\begin{longtable}{|p{5cm}|p{3cm}|p{1cm}|p{4cm}|}
    \hline
    \textbf{Title} & \textbf{Venue} & \textbf{Year} & \textbf{Links} \\
    \hline
    A literature review of supervised and unsupervised machine learning approaches for classification of extended extragalactic objects in astronomical survey data & Eastern Washington University Research Methods & 2025 & \href{https://github.com/dsergio-thesis/.github/blob/main/papers/Literature_Review.pdf}{Paper}  \\
    \hline
    Identifying galaxies, quasars, and stars with machine learning: A new catalogue of classifications for 111 million SDSS sources without spectra& Journal Astronomy and Astrophysics & 2020 & \href{https://github.com/dsergio-thesis/.github/blob/main/presentations/Presentation_Clarke2020.pdf}{Presentation} \\
    \hline
\end{longtable}

\subsection*{Repositories}

\begin{longtable}{|p{4cm}|p{7cm}|p{2cm}|}
    \hline
    \textbf{Repository} & \textbf{Description} & \textbf{Status} \\
    \hline
    repo-1 & Data ingestion, cutout generation, and preprocessing workflows & Active \\
    \hline
    repo-2 & CNN/Transformer models for galaxy morphology & Active \\
    \hline
    repo-3 & Multi-survey catalog cross-matching utilities & Experimental \\
    \hline
\end{longtable}

\subsection*{Contributions}

We welcome contributions from researchers, students, and collaborators interested in:

\begin{itemize}
    \item Machine learning for astronomical imaging
    \item Survey-scale data engineering
    \item Cross-survey validation
    \item Reproducible scientific workflows
    \item Open scientific software
\end{itemize}

Please see individual repositories for contribution guidelines and development standards.

\subsection*{Acknowledgements}

This work builds upon data and infrastructure from major astronomical surveys and open-source scientific ecosystems. We gratefully acknowledge:

\begin{itemize}
    \item Survey teams and data providers
    \item The open-source Python scientific computing community
    \item Collaborating researchers and students
\end{itemize}

\end{document}
