
\documentclass[11pt]{article}
\usepackage[margin=1in]{geometry}
\usepackage{longtable}
\usepackage{array}
\usepackage{hyperref}
\usepackage{titlesec}
\usepackage{graphicx}


\titleformat{\section}{\large\bfseries}{}{0em}{}

%\title{\textbf{Machine Learning in Astronomy}}
%\date{}

\begin{document}

\begin{center}
    \includegraphics[width=0.25\textwidth]{EWU.png}
\end{center}
%\maketitle

\section*{Machine Learning in Astronomy}

A study of machine learning applications in sky surveys, reproducible research, and open scientific collaboration.

This organization develops methods and infrastructure for applying modern machine learning techniques to large-scale astronomical survey data, including imaging, catalogs, and cross-matched multi-survey datasets.

\subsection*{Papers \& Presentations}

\begin{longtable}{|p{5cm}|p{3cm}|p{1cm}|p{4cm}|}
    \hline
    \textbf{Title} & \textbf{Venue} & \textbf{Year} & \textbf{Links} \\
    Classification of Extended Extragalactic Objects in LSST Survey Data using Supervised Machine Learning & Eastern Washington University & 2026 &  \\
    \hline
    A literature review of supervised and unsupervised machine learning approaches for classification of extended extragalactic objects in astronomical survey data & Eastern Washington University Research Methods & 2025 & \href{https://github.com/dsergio-thesis/.github/blob/main/papers/Literature_Review.pdf}{Paper}  \\
    \hline
\end{longtable}

\subsection*{Datasets}

\begin{longtable}{|p{4cm}|p{5cm}|p{2cm}|p{2cm}|p{3cm}|}
\hline
    \textbf{Title} & \textbf{Description} & \textbf{Platform} & \textbf{Size (samples)} \textbf{Link} \\
\hline
    LSST Photometry / HST SFR & LSST FITS-based dataset. This dataset is covered in detail \href{https://github.com/dsergio-thesis/.github/wiki/Datasets}{here}. & Kaggle & 1118 & \href{https://www.kaggle.com/datasets/dsergio/lsst-images-photometry-hst-sfr-labels-1118-samples}{Example LSST/HST Labeled Dataset} \\
\hline
\end{longtable}

\subsection*{Models}

\begin{longtable}{|p{4cm}|p{5cm}|p{2cm}|p{3cm}|}
\hline
\textbf{Title} & \textbf{Description} & \textbf{Platform} & \textbf{Link} \\
\hline
Multi-branch Model & The multi-branch ensemble model. & Hugging Face &  \\
\hline
\end{longtable}

\subsection*{Repositories}

\begin{longtable}{|p{4cm}|p{7cm}|p{2cm}|}
    \hline
    \textbf{Repository} & \textbf{Description} & \textbf{Status} \\
    \hline
    \href{https://github.com/dsergio-thesis/notebooks}{notebooks} & Example notebooks & Active \\
    \hline
    pipelines & Constructing massive datasets at scale. This repository uses the Rubin Science Platform (RSP) and the LSST Stack. & Active \\
    \hline
    models & PyTorch Machine Learning Models & Active \\
    \hline
    astroos-web-app & Web interface for visualizations & Active \\
    \hline
\end{longtable}

\subsection*{Contributions}

We welcome contributions from researchers, students, and collaborators interested in:

\begin{itemize}
    \item Machine learning for astronomical imaging
    \item Survey-scale data engineering
    \item Cross-survey validation
    \item Reproducible scientific workflows
    \item Open scientific software
\end{itemize}

Please see individual repositories for contribution guidelines and development standards.

\subsection*{Acknowledgements}

This work builds upon data and infrastructure from major astronomical surveys and open-source scientific ecosystems. We gratefully acknowledge:

\begin{itemize}
    \item The LSST Survey teams and data providers
    \item The open-source Python scientific computing community
    \item Eastern Washington University collaborating researchers and students
    \item Special thanks to Dr. Chris Cain for advising and supporting this research
\end{itemize}

\end{document}
